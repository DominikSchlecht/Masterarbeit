\chapter{Fazit}
In dieser Arbeit wurden anfangs die verschiedenen Arten von Pen-Tests mit deren jeweiligen Charakteristika und Besonderheiten aufgezeigt. Anschließend wurden die Prozesse zu Pen-Tests, aufgeteilt nach Vorbereitung, Durchführung, Nachbereitung und Kontinuität, vorgestellt und Best-Practices erläutert.\\

Dabei wurden bestehende Praktiken um die Anwendung auf mobile Applikationen erweitert. Zudem wurden für aufwändige Prozessschritte Tools entwickelt, welche die Effizienz sowohl auf Seiten der Pen-Tester als auch des Unternehmens erhöhen können. Diese Schritte sind die Fragebögen zur Aufwandsschätzung und des Kickoffs, sowie die Erstellung des Pen-Test-Reports. Die Inhalte für diese Schritte können effizient und komfortabel in einer Weboberfläche eingepflegt und als PDF-Dokument exportiert werden. In Hinsicht auf mobile Anwendungen wurde anschließend auf Tools und Frameworks eingegangen und das \textit{Mobile Security Framework} (\textit{MobSF}) so weiterentwickelt, dass dieses alle gesetzten Anforderungen erfüllt. Dabei wurden über 3700 Zeilen an Code hinzugefügt oder überarbeitet, um eine genauere Analyse von \textit{iOS}-App und eine statische Analyse von \textit{Windows-Phone}-Apps zu ermöglichen. \\

Abschließend wurde das Framework anhand von zwei Beispielen vorgestellt. Dabei wurden eine \textit{iOS}-App sowie eine \textit{Windows-Phone}-App analysiert und die Ergebnisse dargestellt. Alle entwickelten Werkzeuge werden als Open-Source-Produkte auf Github veröffentlicht.\\

Somit stellt diese Arbeit neu entwickelte oder weiterentwickelte Werkzeuge für den Test von mobilen Applikationen kostenfrei bereit. Diese können Unternehmen sowie private Entwickler nutzen, um auf die stark Ansteigende Nachfrage für mobilen Applikationen zu reagieren und die damit verbundenen sicherheitstechnischen Herausforderungen effizienter zu bewältigen.


