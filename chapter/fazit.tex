\chapter{Fazit}
In dieser Arbeit wurden Anfangs die verschiedenen Arten von Pen-Tests mit deren jeweiligen Charakteristika und Besonderheiten aufgezeigt. Anschließend wurden die Prozesse zu Pen-Tests, aufgeteilt nach Vorbereitung, Durchführung, Nachbereitung und Kontinuität, vorgestellt und Best-Practices erläutert. Zudem wurden für manche Prozessschritte Tools entwickelt, welche die Effizienz sowohl auf Seiten der Pen-Tester als auch des Unternehmens erhöhen können. Daraufhin wurde auf Tools und Frameworks eingegangen und eine bestimmte Anwendung so weiterentwickelt, dass diese alle gesetzten Anforderungen erfüllt. Abschließend wurde das Framework anhand von 2 Beispielen vorgestellt.

TODO mehr