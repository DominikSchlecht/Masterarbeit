\section{Weiterentwicklung MobSF}
Ein Kernelement dieser Arbeit ist die Weiterentwicklung des Mobile Security Frameworks \textit{MobSF}. Die Funktionen des Frameworks sind bereits unter TODO aufgezeigt. Im Folgenden sind die Änderungen dargestellt, welche an dem Framework vorgenommen und veröffentlicht wurden.

\subsection{Syntax}
Bei der Neu- oder Reimplementierung wurde auf die Verwendung von offiziellen Style-Guides geachtet. Insbesondere wurde der PEP 8 Standard \footnote{\url{https://www.python.org/dev/peps/pep-0008/}} für Python verwendet, welcher die Lesbarkeit und Wartbarkeit von Python-Code verbessern soll.

\subsection{Struktur}
Die Struktur von MobSF war bisher relativ flach. Auf der ersten Ebene findet man die Übergeordneten Module wie den ApiTester, StaticAnalyzer, DynamicAnalyzer sowie den statischen Content, Templates und Kern-Module von MobSF. Dies ist in der Abbildung \ref{fig:MobSFStrukOrig} verdeutlicht. Jedoch hatte die Struktur in den Modulen oft keine saubere Trennung der Aufgaben. So waren im \textit{StaticAnalyzer}-Modul sowohl iOS wie auch Android-Analyse in der \textit{views.py} zusammengefasst. Um hier eine klarere Trennung zu schaffen, wurde die \textit{views.py} aufgegliedert in drei Dateien:
\begin{description}
	\item[shared\_func.py: ] Die \textit{shared\_func.py} enthält alle Funktionen, welche sowohl für IOS als auch Android gebraucht werden. Beispiele sind die Erstellung von Hashes oder das Generieren von PDFs.
	
	\item[ios.py: ] Die Datei \textit{ios.py} enthält alle iOS spezifischen Funktionen zur statischen Analyse.
	
	\item[android.py: ] Die Datei \textit{android.py} enthält alle Android spezifischen Funktionen zur statischen Analyse.
\end{description}
Dies ist im Detail in der Abbildung \ref{fig:MobSFStaticStrucVergl} dargestellt.

\begin{figure}
\dirtree{%
.1 Mobile-Security-Framework-MobSF/.
    .2 .git/.
    .2 APITester/.
    .2 downloads/.
    .2 DynamicAnalyzer/.
    .2 LICENSES/.
    .2 logs/.
    .2 MalwareAnalyzer/.
    .2 MobSF/.
    .2 static/.
    .2 StaticAnalyzer/.
    .2 templates/.
    .2 uploads/.
}
\label{fig:MobSFStrukOrig}
\caption{Strutkur MobSF auf der ersten Ebene}
\end{figure}

\begin{figure}
\centering
\begin{subfigure}{.5\textwidth}
  \dirtree{%
.1 StaticAnalyzer/.
    .2 [...].
    .2 views.py.
    	.3 key
    	.3 PDF
    	.3 Java.
    	.3 Smali.
    	.3 Find.
    	.3 ViewSource.
    	.3 ManifestView.
    	.3 StaticAnalyzer.
    	.3 GetHardcodedCertKeystore.
    	.3 ReadManifest.
    	.3 GetManifest.
    	.3 ValidAndroidZip.
    	.3 HashGen.
    	.3 FileSize.
    	.3 GenDownloads.
    	.3 zipdir.
    	.3 Unzip.
    	.3 FormatPermissions.
    	.3 CertInfo.
    	.3 WinFixJava.
    	.3 WinFixPython3.
    	.3 Dex2Jar.
    	.3 Dex2Smali.
    	.3 Jar2Java.
    	.3 Strings.
    	.3 ManifestData.
    	.3 ManifestAnalysis.
    	.3 CodeAnalysis.
    	.3 StaticAnalyzer\_iOS.
    	.3 ViewFile.
    	.3 readBinXML.
    	.3 HandleSqlite.
    	.3 iOS\_ListFiles.
    	.3 BinaryAnalysis.
    	.3 iOS\_Source\_Analysis.
}
  \caption{Alte Struktur}
  \label{fig:sub1}
\end{subfigure}%
\begin{subfigure}{.5\textwidth}
  \dirtree{%
.1 Mobile-Security-Framework-MobSF/.
    .2 [...].
    .2 views/.
		.3 android.py.
	    	.4 key.
	    	.4 Java.
	    	.4 Smali.
	    	.4 Find.
	    	.4 ViewSource.
	    	.4 ManifestView.
	    	.4 StaticAnalyzer.
	    	.4 GetHardcodedCertKeystore.
	    	.4 ReadManifest.
	    	.4 GetManifest.
	    	.4 ValidAndroidZip.
	    	.4 FileSize.
	    	.4 GenDownloads.
	    	.4 zipdir.
	    	.4 Unzip.
	    	.4 FormatPermissions.
	    	.4 CertInfo.
	    	.4 WinFixJava.
	    	.4 WinFixPython3.
	    	.4 Dex2Jar.
	    	.4 Dex2Smali.
	    	.4 Jar2Java.
	    	.4 Strings.
	    	.4 ManifestData.
	    	.4 ManifestAnalysis.
	    	.4 CodeAnalysis.
		.3 ios.py.
	    	.4 StaticAnalyzer\_iOS.
	    	.4 ViewFile.
	    	.4 readBinXML.
	    	.4 HandleSqlite.
	    	.4 iOS\_ListFiles.
	    	.4 BinaryAnalysis.
	    	.4 iOS\_Source\_Analysis.
		.3 shared\_func.py.
	    	.4 key.
	    	.4 FileSize.
	    	.4 HashGen.
	    	.4 Unzip.
	    	.4 PDF.
}
  \caption{Neue Struktur}
  \label{fig:sub2}
\end{subfigure}
\caption{Vergleich der Struktur von StaticAnalyzer}
\label{fig:MobSFStaticStrucVergl}
\end{figure}

\subsection{\textit{strings} bei iOS-Analyse}
Zuerst wurde das MobSF um die Fähigkeit erweitert, eine iOS-Applikation mit dem \textit{strings}-Programm zu untersuchen. \textit{strings} durchsucht, insofern keine zusätzlichen Parameter übergeben werden, eine binäre Datei mach 4 aufeinander folgende ASCII-Elemente.

Dies hilft oft bei einer ersten Einschätzung der Anwendung, da oft eine grundlegende Funktionsweise und der Zweck der Software abgeleitet werden kann. Ebenso können eventuell unbeabsichtigt im Programm vergesse Strings in einer App aufgedeckt werden.

\subsection{Statische Analyse von Windows-Phone-Apps}
Um die drei marktführenden mobilen Betriebssysteme mit MobSF abzudecken, wurden Funktionen für Windows-Phone-Apps hinzugefügt.