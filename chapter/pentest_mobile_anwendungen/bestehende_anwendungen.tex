\section{Bestehende Anwendungen}
Trotz der relativ neuen Thematik der Mobilen Applikationen gibt es schon einige Programme und Applikationen, die bei der Identifizierung von Schwachstellen helfen können. Im Folgenden sind diese unterteilt in \textit{All-In-One-Framework} und Einzelanwendungen. Die Namen sind hierbei sprechend: Sogenannte \textit{All-In-One-Frameworks} bündeln mehrere kleine Anwendungen und automatisieren den Ablauf oder vereinfachen die Bedienung.

\subsection{All-In-One-Framework: MobSF}
\textit{MobSF} ist das einzige, derzeit öffentlich Verbreitete All-In-One-Framework zur Analyse von Mobilen Applikationen. Es ist eine Plattform zur statischen Analyse von Android und iOS-Apps sowie zur dynamischen Analyse von Android Apps. Es bündelt viele kleinere Anwendungen, welche unter \ref{Pen:Eingelanwendungen} aufgeführt sind, in einer einfachen Weboberfläche. Es ist Open-Source, in \textit{Python} geschrieben und steht in \textit{GIT} frei zur Verfügung.\footnote{\url{https://github.com/ajinabraham/Mobile-Security-Framework-MobSF}} Die aktuelle Version ist \textit{0.9.2 beta}.

Es unterstützt die statische Analyse von Apps in den Formaten \textit{APK} und \textit{IPA} sowie aus einfach komprimierten Archiven (\textit{zip}). Zusätzlich beinhaltet \textit{MobSF} einen eingebauten API Fuzzer und ist in der Lage, API-spezifische Schwachstellen wie XXE, SSRF oder Path Traversal zu erkennen (TODO Auflisten).

\subsection{Einzelanwendungen}\label{Pen:Eingelanwendungen}
Das All-In-One-Framework MobSF greift im Hintergrund oft auf eigenständige Tools zurück. Da es für Penetration-Test oft hilfreich ist, diese ohne ein umgebendes Framework nutzen zu können, sind im Folgenden die wichtigsten Tools kurz aufgeführt.

\subsubsection{•}