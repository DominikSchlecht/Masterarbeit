\chapter{Einleitung}
Smartphones verbreiten sich immer stärker. Dies Zeigt auch eine Statistik von Gardner, nach welcher die Absätze von Mobilen Geräten die von herkömmlichen Rechnern und Laptops weiter übertreffen werden \cite{GartnerSales}, siehe Tabelle \ref{ref:GartnerSalesTable}. Mit diesem Fortschritt im Bereich der Verkäufe steigt natürlich auch die Nutzung von Smartphones. Diese können genutzt werden um im Internet zu "surfen", Medien-Inhalte wiederzugeben oder zu Chatten. Dies sind im Bezug auf Datenschutz und Risiko relativ ungefährliche Anwendungen. Jedoch können auch kritischere Handlungen vollzogen werden, wie zum Beispiel Online-Banking oder das Verwalten von Versicherungs-Verträgen. Das dies gemacht wird, ist eine notwendige Konsequenz aus der Natur des Smartphones und deren User. So ist es einfach praktisch, seinen Kontostand schnell von unterwegs einsehen zu können. Jedoch müssen solche Apps dann so gestaltet sein, dass ein Missbrauch nicht oder nur mit unverhältnismäßig hohem Aufwand möglich ist. Dies ist die Aufgabe der Unternehmen, welche solche Apps bereit stellen. Dies stellt viele vor unerwartet hohe Herausforderungen. Mussten bisher nur lokale oder Web-Anwendungen getestet werden, sind es nun Applikationen auf mobilen Geräten mit einem oft wesentlich höheren Anteil an Web-Services und APIs.

\begin{figure}[htbp]
	\begin{tabular}{ l r r r r}
		Device Type & 2015 & 2016 & 2017 & 2018 \\ \hline
		Traditional PCs (Desk-Based and Notebook) & 244 & 228 & 223 & 216 \\
		Ultramobiles (Premium) & 45 & 57 & 73 & 90 \\
		PC Market & 289 & 284 & 296 & 306 \\
		Ultramobiles (Basic and Utility) & 195 & 188 & 188 & 194 \\
		Computing Devices Market & 484 & 473 & 485 & 500 \\
		Mobile Phones & 1,917 & 1,943 & 1,983 & 2,022 \\
		Total Devices Market & 2,401 & 2,416 & 2,468 & 2,521 \\
	\end{tabular}
	\label{ref:GartnerSalesTable}
	\caption{Worldwide Devices Shipments by Device Type, 2015-2018 (Millions of Units)\cite{GartnerSales}}
\end{figure}

Die Allianz Deutschland AG ist ein solches Unternehmen. Bisher waren viele Wege zum Kunden analog, also per Brief. Es gab nicht viele elektronische Schnittstellen. Vor 2 Jahren wurde eine Digitalisierungs-Strategie gegenüber beschlossen und die Web-Anwendung "Meine-Allianz" entwickelt. In dieser können Versicherungsnehmer nicht nur Verträge anschauen, sondern auch Änderungen an diesen Vornehmen. Sollte diese Anwendung eine schwerwiegende Schwachstelle aufweisen, hätten dies für die Allianz nicht nur Schäden in Bezug auf die Reputation, sondern eventuell zusätzliche rechtliche Folgen, da Krankendaten unter §203 StGB fallen. Um dies zu verhindern, sind Prozesse in der Anwendungsentwicklung notwendig, welche die Sicherheit einer Anwendung sicherstellen.

Diese Prozesse umfassen Komponenten wie Source-Scanning, Penetrations-Test sowie eine regelmäßig wiederkehrende Prüfung von Anwendungen, selbst nach dem Release.

In dieser Arbeit werden als Hinleitung allgemeine Prozesse und Fakten rund um die Applikations-Sicherheit erläutert. Im weiterführenden Teil wird ein bereits bestehendes Open-Source-Tool zum automatischen Testen mobiler Anwendungen um Features wie TODO erweitert.

