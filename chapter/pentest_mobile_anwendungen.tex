\chapter{Penetrationstests mobiler Anwendungen}
	\section{Aktuelle Situation und Vergleich}
		\subsection{IOS}
			\subsubsection{Emulation vs. Hardware}
			XCode
			\subsubsection{Debugging}
			XCode
		\subsection{Windows-Phone}
			\subsubsection{Emulation vs. Hardware}
			VS
			\subsubsection{Debugging}
			VS	
		\subsection{Android}
			Android ist ein Ursprünglich 2003 von der Android, Inc. entwickeltes mobiles Betriebssystem. 2005 wurde es durch Google übernommen und wird seit dem weiterentwickelt. 2015 liegt es bei einem Marktanteil von TODO \%. Aufgrund der Quelloffenheit des Systems wird von vielen Herstellern auf verschiedensten Plattformen genutzt. Jedoch bringt die weitführende Fragmentierung des Betriebssystems auch Nachteile mit sich. So sind in 2015 nur TODO \% der Android-Devices auf einer aktuellen Version.\cite{Drake2014}
			\subsubsection{Android-Studio und SDK}
			Das Android-Studio ist eine umfassende IDE. Sie ermöglicht unter anderem das schnelle Entwickeln und Testen von Apps, sowie die Emulation von beliebigen Android-Versionen. Außerdem ist Android Studio kostenlos, Open-Source und für Linux, Mac und Windows erhältlich. Die aktuelle Version kann unter \url{http://developer.android.com/sdk/index.html} heruntergeladen werden. Die Installation unter Linux ist vergleichsweise einfach, da nur ein Archiv über das Kommando 
\begin{lstlisting}
unzip android-studio-ide-143.2739321-linux.zip
\end{lstlisting}
entpackt werden muss. Für alle anderen Betriebssysteme werden entsprechende Installationsroutinen zur Verfügung gestellt. Anschließend kann die IDE über die Datei "`bin/studio.sh"' gestartet werden. Neben dem Android-Studio gibt es noch das Android SDK, welches über die gleiche URL heruntergeladen werde kann. Es enthält wichtige Kommandozeilen-Tools wie \textit{adb}, \textit{fastboot} oder \textit{logcat}, auf welche im weiteren Verlauf noch detailliert eingegangen wird.
			\subsubsection{Compatibility Testing Suite}
			\cite{Drake2014} Seite 18
			\subsubsection{Emulation vs. Hardware}
			Android SDK
			\subsubsection{Debugging}	
			Android Debug Bridge\cite{androidDebugBridge}
			\subsubsection{Logcat}	
			Android Debug Bridge\cite{androidDebugBridge}
	\section{Anforderungen}
	\begin{itemize}
		\item Automatisierung
		\item Blackbox/Whitebox
		\item Reporting
		\item False-Positive-Rate
	\end{itemize}
	\section{Laboraufbau}
	\section{Entwicklung der Umgebung}
		\subsection{Aufbau}
		\subsection{Schnittstellen}
		\subsection{Technisches Detail 1}
		\subsection{Technisches Detail 2}
	\section{Abgleich mit Anforderungen}