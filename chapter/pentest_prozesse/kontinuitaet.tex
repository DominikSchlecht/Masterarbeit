\section{Kontinuität}
Ein weitere wichtiger Teil der Prozesse um Pen-Tests ist Kontinuität. So reicht es nicht, nur einen Pen-Test durchzuführen, wenn eine Anwendung das erste Mal online genommen wird. Denn durch Weiterentwicklung sowie neue entdeckte Schwachstelle kann sich das Sicherheitsniveau ständig ändern.\\

Allgemein gilt, dass das Sicherheitsniveau einer Anwendung immer mindestens der Kritikalität des Anwendung für das Unternehmen entsprechen sollte. Hat eine Anwendung eine hohe Kritikalität, so sollte auch mindestens ein hohes Sicherheitsniveau gelten.\\

Natürlich sollte dafür festgelegt werden, wie die Kritikalität und das Sicherheitsniveau festgelegt sind. Die Kritikalität pro Anwendung ist meist Abhängig vom Unternehmen. So würde für einen Automobilhersteller wohl eine Produktionsanlage als hoch kritisch gesehen, bei einem Online-Vertrieb wohl eher der Online-Shop.\\

Auch das Sicherheitsniveau sowie dessen Aufrechterhaltung muss im Unternehmen definiert werden. Dabei sollte zuerst Maßnahmen definiert werden. 

\begin{description}
	\item[Pen-Tests] sind eine äußerst effektive, aber kostspielige Maßnahme.
	
		\item[Code-Audits] sind von Personen ausgeführte Analysen des Quellcodes auf Schwachstellen. Das das Personal äußerst gut geschult sein muss, sollten hierfür entweder im Unternehmen Experten eingestellt oder extern gebucht werden. Beim Einkauf von externen Dienstleitern, sind die Audits ähnlich teuer wie Pentests, sind aber ebenfalls eine sehr effektive Maßnahme.
	
	\item[Security-Scanner] meint automatisierte Sicherheits-Scans durch Software wie \textit{IBM Security AppScan}\footnote{\url{http://www-03.ibm.com/software/products/de/appscan}} oder der \textit{Nessus Vulnerability Scanner}\footnote{\url{https://www.tenable.com/products/nessus-vulnerability-scanner}}. Security-Scanner sind nicht so genau wie Pen-Tests, dafür sind diese, aber einer gewissen Anzahl von Anwendungen, wesentlich günstiger.
	
	\item[Sourcecode-Scans] ebenso wie Security-Scanner sind Sourcecode-Scanner Produkte, welche automatisiert Analysen durchführen. Produkte wären Beispielsweise \textit{Fortify Static Code Analyzer}\footnote{\url{http://www8.hp.com/de/de/software-solutions/static-code-analysis-sast/}} oder \textit{Veracode Static Analysis}\footnote{\url{https://www.veracode.com/products/binary-static-analysis-sast}}. Sourcecode-Scans sind ähnlich effektiv wie Security-Scanner.
	
\end{description}
 
Hat ein Unternehmen die passenden Maßnahmen aufgebaut, sollten diese über eine zeitliche Einteilung den Sicherheitsniveaus zugeordnet werden. Ein frei gewähltes Beispiel dafür ist in Tabelle \ref{tab:PenKontinuitaet} zu sehen. Dabei ist Jahr 1 das erste Jahr, nachdem die Anwendung (nach einem Pen-Test) online genommen wurde.

\begin{table}
	\begin{tabularx}{\textwidth}{l||p{2.8cm}|p{2.8cm}|p{2.8cm}|p{2.8cm}}
		& Jahr 1 & Jahr 2 & Jahr 3 & Jahr 4 \tabularnewline \hline\hline
	Sehr Hoch  & Pen-Test & Pen-Test & Pen-Test & Pen-Test \tabularnewline \hline
	Hoch  & Code-Audit \newline Security-Scan & Pen-Test & Code-Audit & Pen-Test \newline Security-Scan \tabularnewline \hline
	Mittel  & Security-Scan \newline Sourcecode-Scan & Code-Audit & Security-Scan \newline Sourcecode-Scan  & Pen-Test \tabularnewline \hline
	Niedrig  & Security-Scan & Security-Scan & Security-Scan & Security-Scan \tabularnewline \hline
	Sehr Niedrig  & - & Security-Scan & - & Security-Scan \tabularnewline
	\end{tabularx}

	\label{tab:PenKontinuitaet}
	\caption{Einteilung von Maßnahmen zum Sicherheitsniveau über bestimmte Zeiträume}
\end{table}
