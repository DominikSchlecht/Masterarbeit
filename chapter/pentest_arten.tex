\chapter{Arten von Penetration-Tests}
Als Einführung dient zunächst ein kurzer Überblick über die verschiedenen Arten von Penetration-Tests. An diesen orientiert sich ebenfalls das zweite Kapitel, Prozesse zu Penetration-Tests (\ref{ref:prozpentest}). Die Arten basieren dabei auf der Liste von Pentest-Standard.org\footnote{\url{http://www.pentest-standard.org/index.php/Pre-engagement}}, sind jedoch um weitere Arten, basierend auf den Erfahrungen des Autors sowie andere Pen-Tester der Allianz Deutschland AG, ergänzt.

	\section{Web-Application}
	\textit{Web-Application-Tests} sind wohl die häufigste durchgeführten Tests. Dabei wird eine definierte Menge an Webseiten von einem sogenannten Penetration-Tester auf Schwachstellen getestet. Typische Findings im bei \textit{Web-Application-Tests} wären fehlende \textit{Cookie-Flags}, \textit{XSS}, \textit{CSRF} oder auch \textit{SQL-Injection}. Ebenso gehören neben technischen Sicherheitslücken auch Mängel in der Anwendungslogik zur Menge der möglichen Findings.
	
	\section{Web-Service}
	Ebenfalls sehr häufig und immer öfter gefragt sind Pen-Tests auf Web-Services. Diese haben im Gegensatz zu \textit{Web-Applications} keine Oberfläche sondern sind reine Schnittstellen zur Abfrage von Informationen. Zudem sind sie meist in \textit{REST}, \textit{WSDL} oder \textit{SOAP} geschrieben, weshalb sich das Vorgehen bei Pen-Test grundlegend von anderen Arten unterscheidet.
	
	\section{Mobile-Application}
	Von einem Penetration-Test auf eine Mobile-Application redet man dann, wenn eine sog. App, also eine Anwendung geschrieben für ein mobiles Gerät, getestet wird. Da der Begriff nicht weiter definiert ist (TODO), ist unklar, ob ein Mobile-Application-Pen-Test ebenfalls die Backend-Systeme einschließt. Der Unterschied wäre dabei, dass bei einem Test ohne Backend-Tests nur der Teil der App auf dem Handy untersucht wird. Hier können Lücken wie \textit{Buffer-Overflows}, mangelnde \textit{TLS}-Prüfungen und ähnliches gefunden werden, welche eher den Benutzer eine Mobiltelefons gefährden. Test auf die Backend-Systeme von Apps dagegen sollen eher den Betreiber der Applikation schützen. Eine Lücke wäre hier beispielsweise eine SQL-Injektion, durch welche ein Angreifer im schlimmsten Fall alle Daten des Betreibers abfragen oder auch Code auf dem DB-System ausführen könnte.
	
	\section{Network}
	Bei einem \textit{Network-Pen-Test} geht es zumeist darum, in eine Netzwerk einzubrechen, die Bestandteil zu erfassen und anschließend zu analysieren. Oft werden auch bestimmte Systeme als Ziele definiert, welche es zu übernehmen gilt. Typische Findings wären veraltete Systeme oder unerwünschte Komponenten im Netzwerk (sogenannte "`Schatten-IT"'). Da diese Test meist in der produktiven Umgebung durchgeführt werden, sollte zur Verhinderung von Störungen im Betrieb vorher definiert werden, welche Systeme vom Test ausgenommen sind.
	
%	\section{Krypographie}
%	Bei einem Test von kryptographischen Systemen ist meist die Zielstellung, die Sicherheits-Funktion eines Algorithmus zu umgehen. Im Beispiel eines verschlüsselungs-Algorithmus wäre dies das Auffinden einer Schwachstelle, die eine schnelle Entschlüsselung erlaubt. Ein anderes praktisches Beispiel wäre zum Beispiel ein Test, ob bei einer Web-Anwendung Session-IDs einfach zu berechnen sind.
	
	\section{Wireless Network}
	Bei einem Test von Wireless-Systemen meint man den Angriff auf IT-Systeme, welche Funktechniken nutzen. Oft wird hier die  WLAN-Konfiguration eines Unternehmens daraufhin getestet, ob ein Eindringen in das Netzwerk oder das Abfangen von Daten möglich ist. In den letzten Jahren sind jedoch auch Tests auf andere Funktechniken wie NFC/RFID häufiger geworden, seitdem diese oft als Zugangskontrollen oder Zahlungsmittel in Unternehmen verwendet werden. Ebenso sind hier Tests auf Funk-Systeme im 433 oder 900 MHz Bereich angesiedelt, welche häufig bei z.B. Garagentoren oder Auto-Schlüsseln verwendet werden.
	
	\section{Social Engineering}
	Der BSI-Grundschutz beschreibt \textit{Social Engineering} wie folgt:\footnote{\url{https://www.bsi.bund.de/DE/Themen/ITGrundschutz/ITGrundschutzKataloge/Inhalt/_content/g/g05/g05042.html}}
	\begin{quote}
Social Engineering ist eine Methode, um unberechtigten Zugang zu Informationen oder IT-Systemen durch "Aushorchen" zu erlangen. Beim Social Engineering werden menschliche Eigenschaften wie z. B. Hilfsbereitschaft, Vertrauen, Angst oder Respekt vor Autorität ausgenutzt. 
	\end{quote}

Ein Beispiel wäre, dass ein Pen-Tester versucht Zugang zu einem Gebäude der Firma zu erlangen, indem er eine Blumenlieferung an einen Vorstand vortäuscht. Dabei stehen ihm sowohl die Möglichkeit der Autorität ("`Ich darf die Blumen nur direkt dem Vorstand liefern. Wenn Sie mich nicht durchlassen, wird er das erfahren."') als auch der Hilfsbereitschaft ("`Wenn ich diese Lieferung nicht abschließe, werde ich gefeuert."') zur Verfügung.

	\section{Physical}
	Bei einem \textit{Physical-Pen-Test} versteht man den Versuch, physikalische Sicherheitsmaßnahmen zu überwinden. Diese werden oft in Verbindung mit \textit{Social Engineering} verwendet und sind oft teil eines \textit{Redteam}-Tests. Da hier bei Sicherheitskomplexen mit zum Beispiel bewaffnetem Wachpersonal ein besonderes Risiko für die Tester vorliegt, sollten sie besonders sorgfältig vorbereitet werden.

	\section{Redteam}
	Bei einem Redteam/Blueteam wird den Testern, welche das sogenannte Redteam bilden, im Gegensatz zu den anderen Arten von Penetration-Tests keine bestimmte Menge an Webseiten oder Angriffspunkten genannt, sondern nur ein bestimmtes, zu erreichendes Ziel. Dieses können bestimme Daten sein, wie zum Beispiel die E-Mails eines Vorstands, oder die Kompromittierung eines bestimmten Systems, wie zum Beispiel dem \textit{Active Directory}. Je nach Vereinbarung werden von den Testern Variationen der vorherigen Arten von Angriffen genutzt, um an das definierte Ziel zu gelangen. Dem Redteam gegenüber steht das Blueteam, meist ein SOC oder CERT der Firma, welches versucht die Angriffe zu detektieren und vereiteln. Das Blueteam kann über den bevorstehenden Test informiert werden, in der Praxis wird darauf jedoch bewusst verzichtet, um realitätsnahe Ergebnisse zu erziehen.\\
	

	
	
	
	
	